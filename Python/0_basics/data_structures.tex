\documentclass[knowledge.tex]{subfiles}


\rhead{\textbf{\textit{Python Podstawy - Struktury Danych}}}

\begin{document}
    \chapter{Struktury Danych}
    W Pythonie dostępnych jest kilka złożonych typów danych, które służą do zbierania różnych wartości. Nazywamy je strukturami danych. Podczas kursu \textit{Python Podstawy} nie będziemy implementować od zera tych struktur, lecz poznamy różnice działania między nimi oraz nauczymy się korzystać z ich gotowych implementacji w bibliotece standardowej. 
    \section{Lista}
    Wyobraźmy sobie listę na przykładzie listy zakupów do sklepu. Myślę, że każdy z nas przynajmniej raz w życiu utworzył taką listę, bądź dostał ją od kogoś wybierając się na zakupy. Na kartce z reguły taka lista jest zapisywana z góry do dołu, natomiast podczas przygody z programowaniem częściej będziemy się spotykać z zapisem kolejnych elementów listy w poziomie. Wybierzmy się w takim razie do sklepu i zbudujmy listę zawierającą cztery napisy przedstawiające co mamy kupić. Kolejno mleko, banany, wodę oraz ser.\\[0.3cm]
    W celu odwołania się do elementu znajdującego się na liście będziemy posługiwać się \textit{indeksowaniem}. Lista jest sekwencją wartości, w której zachowana jest kolejność. Pierwszy element jaki znajduje się na liście czyli mleko znajduje się pod indeksem równym 0. Pod indeksem 1 znajduje się napis banany, i tak dalej. Ostatni element znajduje się pod indeksem równym długości listy minus jeden, czyli w rozważanym przypadku pod indeksem 3.\\[0.3cm]
    Listę w Pythonie tworzymy za pomocą nawiasów kwadratowych. Pustą listę buduje się tylko za pomocą nawiasów kwadratowych. W przypadku, kiedy chcemy utworzyć listę zawierającą już konkretne elementy, muszą one znaleźć się wewnątrz nawiasów kwadratowych oddzielone przecinkami. Listy w Pythonie są heterogeniczne (w innych językach może być inaczej), to znaczy, że mogą mogą zawierać elementy różnych typów, ale w większości naszych przykładów wszystkie wartości znajdujące się na liście będą o takim samym typie.\\[0.3cm]
    W Pythonie odwołujemy się do elementów znajdujących się w liście za pomocą nawiasów kwadratowych zawierających indeks elementu, który chcemy odczytać wewnątrz.\\[0.3cm]
    W celu dodania elementu na koniec istniejącej już listy należy skorzystać z funkcji append wywołanej na liście. Przyjmuje ona jeden argument: element, który zamierzamy dodać do listy. Do listy można przypisywać wartości elementów pod wskazany indeks za pomocą znaku równości. Jednakże może to uczynić tylko dla indeksów, które już istnieją na liście. Nie można nadpisać elementu, który nie istnieje.
    
    
    \section{Krotka}
    \section{Słownik}
    a
\end{document}