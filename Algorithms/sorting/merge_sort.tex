\documentclass[knowledge.tex]{subfiles}


\begin{document}
    \section{Merge Sort}
    Merge Sort jest jak drugim obok Quick Sort szybkim algorytmem sortowania z którym się zapoznamy. Powstał nieco wcześniej, był powszechnie wykorzystywany, po czym został wyparty przez QuickSort. Opracowany został przez prekursora dzisiejszej informatyki John von Neumann już w roku 1945.\\[0.3cm]
    Merge sort to także algorytm z rodziny \textbf{Dziel i Zwyciężaj}. Jego idea jest w zasadzie prosta. Pomysłem von Neumanna było podzielenie sortowanej tablicy na pół, posortować rekurencyjnie obydwie a na końcu połączyć je w posortowany sposób.\\[0.3cm]
    \subsection*{Łączenie posortowanych list}
    Zanim zajmiemy się samym sortowaniem zaczniemy od algorytmu, który jest klasyką rozmów kwalifikacyjnych i prawdopodobnie jednym z najczęściej zadawanych zadań. Polega na połączeniu dwóch posortowanych list wejściowych w posortowaną listę wynikową tj:
    \begin{center}
        $(1, 5, 8) + (0, 2, 3, 4, 7, 9) \rightarrow (0,1,2,3,4,5,7,8,9)$
    \end{center}
    % \subsection*{Przykład krok po kroku}
    % Weźmy pod uwagę listę liczb $(1, 8, 3, 9, 4, 5, 7)$ i posortujmy ją od najmniejszej do największej wartości przy użyciu Merge Sort. 
    
    \begin{enumerate}
        \item $(\textbf{1}, 5, 8) + (\textbf{0}, 2, 3, 4, 7, 9) \rightarrow (0)$
        \item $(\textbf{1}, 5, 8) + (0, \textbf{2}, 3, 4, 7, 9) \rightarrow  (0,1)$
        \item $(1, \textbf{5}, 8) + (0, \textbf{2}, 3, 4, 7, 9) \rightarrow  (0,1,2)$
        \item $(1, \textbf{5}, 8) + (0, 2, \textbf{3}, 4, 7, 9) \rightarrow  (0,1,2,3)$
        \item $(1,\textbf{5}, 8) + (0, 2, 3, \textbf{4}, 7, 9) \rightarrow  (0,1,2,3,4)$
        \item $(1, \textbf{5}, 8) + (0, 2, 3, 4, \textbf{7}, 9) \rightarrow  (0,1,2,3,4,5)$
        \item $(1, 5, \textbf{8}) + (0, 2, 3, 4, \textbf{7}, 9) \rightarrow  (0,1,2,3,4,5,7)$
        \item $(1, 5, \textbf{8}) + (0, 2, 3, 4, 7, \textbf{9}) \rightarrow  (0,1,2,3,4,5,7,8)$
        \item $(1, 5, 8) + (0, 2, 3, 4, 7, \textbf{9}) \rightarrow  (0,1,2,3,4,5,7,8,9)$
    \end{enumerate}
    
    \begin{algorithm}[H]
    \SetKwInOut{Input}{input}
    \SetKwInOut{Output}{output}

    \Input{Two sorted lists, like:   $(1, 5, 8) + (0, 2, 3, 4, 7, 9)$}
    \Output{Merges input list into sorted list: $(0,1,2,3,4,5,7,8,9)$}\\[0.3cm]
    
    quick\_sort(values, i, j)
    \Begin {
    \If {i \textbf{less than} j} {
        index = partition(values, i, j)\\
        quick\_sort(values, i, index - 1)\\
        quick\_sort(values, index + 1, j)
    }
    }
    

    \caption{MergeLists}
\end{algorithm}
    \begin{center}
        \begin{tabular}{|  c | c  c  c  c  c  c  c  c |}
            \hline
           i & & \downarrow &  &  &  &  &  &   \\ 
             & & 1 & 8 & 3 & 9 & 4 & 5 & \underline{7} \\  
           j & \uparrow &  &  &  &  &  &  &   \\
      \hline
      index & -1 & 0 & 1 & 2 & 3 & 4 & 5 & 6\\
      \hline
    \end{tabular}        
    \end{center}
    \textbf{Krok pierwszy}\\[0.3cm]
    Wartość pod wskaźnikiem górnym jest mniejsza od pivota ($1 < 7$), więc przesuwamy wskaźnik dolnym i zamieniamy miejscami wartości między obydwoma wskaźnikami, czyli jedynkę samą ze sobą.
    \begin{center}
        \begin{tabular}{|c | c c c c c c c c|}
            \hline
           i & & \downarrow &  &  &  &  &  &   \\ 
             & & \textbf{1} & 8 & 3 & 9 & 4 & 5 & \underline{7} \\  
           j &  & \uparrow &  &  &  &  &  &   \\
      \hline
      index & -1 & 0 & 1 & 2 & 3 & 4 & 5 & 6\\
      \hline
    \end{tabular}     
    \quad $\rightarrow$~~
    \begin{tabular}{|c | c c c c c c c c|}
            \hline
           i & & \downarrow &  &  &  &  &  &   \\ 
             & & 1 & 8 & 3 & 9 & 4 & 5 & \underline{7} \\  
           j &  & \uparrow &  &  &  &  &  &   \\
      \hline
      index & -1 & 0 & 1 & 2 & 3 & 4 & 5 & 6\\
      \hline
    \end{tabular}
    \end{center}
    \textbf{Krok drugi}\\[0.3cm]
    Ósemka jest większa od siódemki, zatem nie robimy nic.
    \begin{center}
        \begin{tabular}{|c | c c c c c c c c|}
            \hline
           i & &  & \downarrow &  &  &  &  &   \\ 
             & & 1 & 8 & 3 & 9 & 4 & 5 & \underline{7} \\  
           j &  & \uparrow &  &  &  &  &  &   \\
      \hline
      index & -1 & 0 & 1 & 2 & 3 & 4 & 5 & 6\\
      \hline
    \end{tabular}    
    \quad $\rightarrow$~~
    \begin{tabular}{|c | c c c c c c c c|}
            \hline
           i & &  & \downarrow &  &  &  &  &   \\ 
             & & 1 & 8 & 3 & 9 & 4 & 5 & \underline{7} \\  
           j &  & \uparrow &  &  &  &  &  &   \\
      \hline
      index & -1 & 0 & 1 & 2 & 3 & 4 & 5 & 6\\
      \hline
    \end{tabular}
    \end{center}
    \textbf{Krok trzeci}\\[0.3cm]
    Trójka jest mniejsza od siódemki, zatem przesuwamy dolny wskaźnik i zamieniamy miejscami wartości spod obydwu wskaźników.
    \begin{center}
        \begin{tabular}{|c | c c c c c c c c|}
            \hline
           i & &  &  & \downarrow &  &  &  &   \\ 
             & & 1 & 8 & 3 & 9 & 4 & 5 & \underline{7} \\  
           j &  & \uparrow &  &  &  &  &  &   \\
      \hline
      index & -1 & 0 & 1 & 2 & 3 & 4 & 5 & 6\\
      \hline
    \end{tabular}    
    \quad $\rightarrow$~~
    \begin{tabular}{|c | c c c c c c c c|}
            \hline
           i & &  &  & \downarrow &  &  &  &   \\ 
             & & 1 & \textbf{3} & \textbf{8} & 9 & 4 & 5 & \underline{7} \\  
           j &  &  & \uparrow &  &  &  &  &   \\
      \hline
      index & -1 & 0 & 1 & 2 & 3 & 4 & 5 & 6\\
      \hline
    \end{tabular}
    \end{center}
    \textbf{Krok czwarty}\\[0.3cm]
    Dziewiątka jest większa od pivota, więc nie robimy nic.
    \begin{center}
        \begin{tabular}{|c | c c c c c c c c|}
            \hline
           i & &  &  &  & \downarrow &  &  &   \\ 
             & & 1 & 3 & 8 & 9 & 4 & 5 & \underline{7} \\  
           j &  &  & \uparrow &  &  &  &  &   \\
      \hline
      index & -1 & 0 & 1 & 2 & 3 & 4 & 5 & 6\\
      \hline
    \end{tabular}    
    \quad $\rightarrow$~~
    \begin{tabular}{|c | c c c c c c c c|}
            \hline
           i & &  &  &  & \downarrow &  &  &   \\ 
             & & 1 & 3 & 8 & 9 & 4 & 5 & \underline{7} \\  
           j &  &  & \uparrow &  &  &  &  &   \\
      \hline
      index & -1 & 0 & 1 & 2 & 3 & 4 & 5 & 6\\
      \hline
    \end{tabular}    
    \end{center}
    \textbf{Krok piąty}\\[0.3cm]
    Czwórka jest mniejsza od siódemki, zatem przesuwamy dolny wskaźnik w prawo i zamieniamy miejscami wartości spod obydwu wskaźników.
    \begin{center}
        \begin{tabular}{|c | c c c c c c c c|}
            \hline
           i & &  &  &  &  & \downarrow &  &   \\ 
             & & 1 & 3 & 8 & 9 & 4 & 5 & \underline{7} \\  
           j &  &  & \uparrow &  &  &  &  &   \\
      \hline
      index & -1 & 0 & 1 & 2 & 3 & 4 & 5 & 6\\
      \hline
    \end{tabular}    
    \quad $\rightarrow$~~
    \begin{tabular}{|c | c c c c c c c c|}
            \hline
           i & &  &  &  &  & \downarrow &  &   \\ 
             & & 1 & 3 & \textbf{4} & 9 & \textbf{8} & 5 & \underline{7} \\  
           j &  &  &  & \uparrow &  &  &  &   \\
      \hline
      index & -1 & 0 & 1 & 2 & 3 & 4 & 5 & 6\\
      \hline
    \end{tabular}    
    \end{center}
    \textbf{Krok szósty}\\[0.3cm]
    Piątka jest mniejsza od siódemki, zatem przesuwamy dolny wskaźnik w prawo i zamieniamy miejscami wartości spod obydwu wskaźników.
    \begin{center}
        \begin{tabular}{|c | c c c c c c c c|}
            \hline
           i & &  &  &  &  &  & \downarrow &   \\ 
             & & 1 & 3 & 4 & 9 & 8 & 5 & \underline{7} \\  
           j &  &  &  & \uparrow &  &  &  &   \\
      \hline
      index & -1 & 0 & 1 & 2 & 3 & 4 & 5 & 6\\
      \hline
    \end{tabular}    
    \quad $\rightarrow$~~
    \begin{tabular}{|c | c c c c c c c c|}
            \hline
           i & &  &  &  &  &  & \downarrow &   \\ 
             & & 1 & 3 & 4 & \textbf{5} & 8 & \textbf{9} & \underline{7} \\  
           j &  &  &  & & \uparrow  &  &  &   \\
      \hline
      index & -1 & 0 & 1 & 2 & 3 & 4 & 5 & 6\\
      \hline
    \end{tabular}    
    \end{center}
    \textbf{Krok siódmy}\\[0.3cm]
    Górny wskaźnik przeszedł do końca listy. Należy przesunąć dolny wskaźnik w prawo oraz zamienić miejscami pivot z wartością spod dolnego wskaźnika.
    \begin{center}
        \begin{tabular}{|c | c c c c c c c c|}
            \hline
           i & &  &  &  &  &  &  & \downarrow  \\ 
             & & 1 & 3 & 4 & 5 & 8 & 9 & \underline{7} \\  
           j &  &  &  & & \uparrow  &  &  &   \\
      \hline
      index & -1 & 0 & 1 & 2 & 3 & 4 & 5 & 6\\
      \hline
    \end{tabular} 
    \quad $\rightarrow$~~
    \begin{tabular}{|c | c c c c c| c| c c|}
            \hline
           i & &  &  &  &  &  &  & \downarrow  \\ 
             & & 1 & 3 & 4 & 5 & \textbf{7} & 9 & \textbf{8} \\  
           j &  &  &  & &   & \uparrow &  &   \\
      \hline
      index & -1 & 0 & 1 & 2 & 3 & 4 & 5 & 6\\
      \hline
    \end{tabular} 
    \end{center}
    
    \subsection*{Rekursywne wywołanie}
    Po wykonaniu procedury \textit{partiton} mamy pewność, że element początkowo oznaczony jako pivot znajduje się na odpowiednim miejscu w tablicy. Następnie należy wywołać sortownie szybkie na lewej i prawej podtablicy.\\[0.3cm]
    Należy jeszcze zastanowić się w jaki sposób wywoływać algorytm sortowania szybkiego, żeby zapewnić sortowanie w miejscu, tzn bez alokowania dodatkowej pamięci oraz nowych tablic. Rozwiązanie powinna nam sugerować implementacja procedury partition, która wykorzystuje dwa wskaźniki. Procedura \textit{quick\_sort} będzie przyjmować tablicę, którą chcemy posortować oraz dwa indeksy mówiące od którego do którego elementu chcemy posortować tablicę. W pierwszym wywołaniu ich wartości przyjmą 0 oraz $N-1$. Poniżej przykład drzewa wywołań \textit{quick\_sort} dla tablicy rozważanej wcześniej $(1, 8, 3, 9, 4, 5, 7)$
    \begin{center}
        \Tree 
    [
        .$qs((\underline{1,8,3,9,4,5,\textbf{7}}),0,6)$ 
            [.$qs((\underline{1,3,4,\textbf{5}},...),0,3)$ 
                [.$qs((\underline{1,3,\textbf{4}},...),0,2)$ ] 
                [.$qs((...,\underline{\textbf{5}},),3,3)$ ]
            ] 
            [.$qs((...,\underline{9,\textbf{8}}),5,6)$
                [.$qs((...,\underline{\textbf{8}},...),5,5)$ ]
                [.$qs((...,\underline{\textbf{9}}),6,6)$ ]
            ]
    ]    
    \end{center}
    Na podstawie omówionego przykładu można zauważyć warunek kiedy konieczne jest kolejne wywołanie rekurencyjne algorytmu \textit{quick\_sort}.
    
    \begin{definition}[Quick Sort]
    Dopóki lewy indeks jest mniejszy od prawego indeksu wykonaj procedurę partition i rekurencyjnie wywołaj sortowanie na lewo i na prawo od miejsca w którym został umieszczony pivot.
    \end{definition}
    
    \subsection*{Analiza złożoności}
    Zastanówmy się nad złożonością obliczeniową szybkiego algorytmu sortowania.
    
    \subsection*{Pseudokod}
    
    \begin{algorithm}[H]
    \SetKwInOut{Input}{input}
    \SetKwInOut{Output}{output}

    \Input{List of comparable objects of length N. For example list of integers $( 5,4,2,1,8 )$}
    \Output{Ordered input list in ascending order.  $( 1, 2, 4, 5, 8 )$}\\[0.3cm]
    
    quick\_sort(values, i, j)
    \Begin {
    \If {i \textbf{less than} j} {
        index = partition(values, i, j)\\
        quick\_sort(values, i, index - 1)\\
        quick\_sort(values, index + 1, j)
    }
    }
    

    \caption{Quick Sort}
\end{algorithm}

\subsection*{Zadania}
\end{document}