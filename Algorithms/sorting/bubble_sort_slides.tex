\documentclass[kslide.tex]{subfiles}

\begin{document}


\begin{frame}
\frametitle{Bubble Sort}
\begin{block}{}
\justifying
Sortowanie bąbelkowe, to prosty algorytm sortowania, który wielokrotnie przechodzi przez listę, porównuje sąsiadujące elementy i zamienia je, jeśli są w niewłaściwej kolejności.\\[0.3cm]\pause

Przejście przez listę jest powtarzane do momentu posortowania listy.\\[0.3cm]\pause

Algorytm nazywa się bąbelkowym, gdyż w każdym kolejnym przejściu przez listę największy możliwy element \textit{wypływa do góry} jak bąbelek w wodzie.
\end{block}
\end{frame}

\begin{frame}
\frametitle{Bubble Sort - Przykład krok po kroku}
Weźmy pod uwagę listę liczb $( 5, 1, 4, 2, 8 )$ i posortujmy ją od najmniejszej do największej wartości przy użyciu sortowania bąbelkowego. W każdym kroku aktualnie porównywane elementy będę zapisane pogrubioną czcionką.\\[0.3cm]\pause
\textbf{Pierwsze przejście:}\\[0.1cm]
        $( \textbf{5, 1}, 4, 2, 8 ) \rightarrow ( \textbf{1, 5}, 4, 2, 8 )$ Algorytm porównuje piątkę z jedynką, a następnie zamienia je, gdyż piątka jest większa od jedynki.\\[0.1cm]
        \pause
        $( 1, \textbf{5, 4}, 2, 8 ) \rightarrow ( 1, \textbf{4, 5}, 2, 8 )$ Algorytm porównuje piątkę z czwórką, a następnie zamienia je, gdyż piątka jest większa od czwórki.\\[0.1cm]
        \pause
        $( 1, 4, \textbf{5, 2}, 8 ) \rightarrow ( 1, 4, \textbf{2, 5}, 8 )$ Algorytm porównuje piątkę z dwójką, a następnie zamienia je, gdyż piątka jest większa od dwójki.\\[0.1cm]
        \pause
        $( 1, 4, 2, \textbf{5, 8} ) \rightarrow ( 1, 4, 2, \textbf{5, 8} )$ Algorytm porównuje piątkę z ósemką, a następnie pozostawia je w tej samej kolejności, gdyż piątka jest mniejsza od ósemki.\\[0.1cm]
\end{frame}

\begin{frame}
\frametitle{Bubble Sort - Przykład krok po kroku}
\textbf{Drugie przejście:}\\[0.1cm]
        $( \textbf{1, 4}, 2, 5, 8 ) \rightarrow ( \textbf{1, 4}, 2, 5, 8 )$ Algorytm porównuje jedynkę z czwórką, a następnie pozostawia je w tej samej kolejności, gdyż jedynka jest mniejsza od czwórki.\\[0.1cm]
        \pause
        $( 1, \textbf{4, 2}, 5, 8 ) \rightarrow ( 1, \textbf{2, 4}, 5, 8 )$ Algorytm porównuje czwórkę z dwójką, a następnie zamienia je, gdyż dwójka jest mniejsza od czwórki.\\[0.1cm]
        \pause
        $( 1, 2, \textbf{4, 5}, 8 ) \rightarrow ( 1, 2, \textbf{4, 5}, 8 )$ Algorytm porównuje czwórkę z piątką, a następnie pozostawia je w tej samej kolejności, gdyż czwórka jest mniejsza od piątki.\\[0.1cm]
        \pause
        $( 1, 2, 4, \textbf{5, 8} ) \rightarrow ( 1, 2, 4, \textbf{5, 8} )$ Algorytm porównuje piątkę z ósemką, a następnie pozostawia je w tej samej kolejności, gdyż ósemka jest większa od piątki.\\[0.1cm]
\end{frame}

\begin{frame}
\frametitle{Bubble Sort - Przykład krok po kroku}
\textbf{Trzecie przejście:}\\[0.1cm]
        $( \textbf{1, 2}, 4, 5, 8 ) \rightarrow ( \textbf{1, 2}, 4, 5, 8 )$ Algorytm porównuje jedynkę z czwórką, a następnie pozostawia je w tej samej kolejności, gdyż jedynka jest mniejsza od dwójki.\\[0.1cm]
        \pause
        $( 1, \textbf{2, 4}, 5, 8 ) \rightarrow ( 1, \textbf{2, 4}, 5, 8 )$ Algorytm porównuje czwórkę z dwójką, a następnie pozostawia je w tej samej kolejności, gdyż dwójka jest mniejsza od czwórki.\\[0.1cm]
        \pause
        $( 1, 2, \textbf{4, 5}, 8 ) \rightarrow ( 1, 2, \textbf{4, 5}, 8 )$ Algorytm porównuje czwórkę z piątką, a następnie pozostawia je w tej samej kolejności, gdyż czwórka jest mniejsza od piątki.\\[0.1cm]
        \pause
        $( 1, 2, 4, \textbf{5, 8} ) \rightarrow ( 1, 2, 4, \textbf{5, 8} )$ Algorytm porównuje piątkę z ósemką, a następnie pozostawia je w tej samej kolejności, gdyż piątka jest mniejsza od ósemki.\\[0.1cm]
\end{frame}

\begin{frame}
\frametitle{Bubble Sort - Analiza złożoności}
        Zastanówmy się nad złożonością obliczeniową algorytmy sortowania bąbelkowego. 
        \begin{itemize}
            \item<1-> Jednokrotne przejście przez listę o długości $N$ wymaga $N - 1$ porównań.
            \item<2-> Po każdym przejściu przez największy możliwy element przesuwany jest na odpowiednie miejsce.
            \item<3->  Weźmy pod uwagę najgorszy możliwy przypadek, w którym wejściowa lista jest odwrotnie posortowana, czyli jej elementy uporządkowane są w malejącej kolejności.
            \item<4->  Za przykład posłuży nam lista $( 8, 5, 4, 2, 1 )$.
        \end{itemize} 
\end{frame}

\begin{frame}
\frametitle{Bubble Sort - Analiza złożoności}
    \textbf{1:} $( 8, 5, 4, 2, 1 ) \rightarrow ( 5, 4, 2, 1, \textbf{8} )$ Po czterech porównaniach ósemka znalazła się na ostatnim miejscu listy.\\[0.1cm]
    \pause
    \textbf{2:} $( 5,4,2,1,8 ) \rightarrow ( 4, 2, 1, \textbf{5}, 8 )$ Po czterech porównaniach piątka znalazła się na drugim od końca miejscu.\\[0.1cm]
    \pause
    \textbf{3:} $( 4, 2, 1, 5, 8 ) \rightarrow ( 2, 1, \textbf{4}, 5, 8 )$ Po czterech porównaniach czwórka znalazła się na trzecim od końca miejscu.\\[0.1cm]
    \pause
    \textbf{4:} $( 2, 1, 4, 5, 8 ) \rightarrow ( 1, \textbf{2}, 4, 5, 8 )$ Po czterech porównaniach dwójka znalazła się na czwartym od końca miejscu.\\[0.1cm]
    \pause
    \textbf{5:} $( 1, 2, 4, 5, 8 ) \rightarrow ( \textbf{1}, 2, 4, 5, 8 )$ Po czterech porównaniach dwójka znalazła się na czwartym od końca miejscu.
\end{frame}

\begin{frame}
\frametitle{Bubble Sort - Analiza złożoności}

Na podstawie powyższego przykładu, możemy stwierdzić, że w najgorszym możliwym przypadku należy wykonać $N$ przejść przez listę, gdzie w każdym z nich należy wykonać $N - 1$ porównań.\\[0.3cm]

Zatem w celu przeprowadzenia sortowania przy użyciu tego algorytmu należy w najgorszym wypadku wykonać $N * (N - 1)$ porównań co równa się $N^2 - N$ porównań, co wyrażone w notacji dużego $O$ złożoność obliczeniowa tego algorytmu to $O(N^2)$, gdzie $N$ to długość sortowanej listy.
\end{frame}

\begin{frame}
\frametitle{Bubble Sort - Pseudokod}
    \begin{algorithm}[H]
    \SetKwInOut{Input}{input}
    \SetKwInOut{Output}{output}

    \Input{List of comparable objects of length N. For example list of integers $( 5,4,2,1,8 )$}
    \Output{Ordered input list in ascending order.  $( 1, 2, 4, 5, 8 )$}\\[0.3cm]
    
    \For{n \textbf{in} [0; N)} {
        \For{i \textbf{in} [0; N - 1)}{
            \If{values[$i$] \textbf{greater than} values[$i + 1$]}{
                swap(values[$i$], values[$i+1$])
            }
        }
    }
    \caption{Bubble Sort}
\end{algorithm}
\end{frame}

\end{document}