\documentclass[kslide.tex]{subfiles}

\begin{document}
\begin{frame}{Złożoność obliczeniowa }
        Służy do porównywania skomplikowania algorytmów w oderwaniu od środowiska, w którym będą wykonywane.\\[0.3cm]\pause 
        
        Mówi o tym, jak koszt wykonania algorytmu rośnie w zależności od rozmiaru danych wejściowych.\\[0.3cm]\pause 
        
        Może opisywać czas wykonania oraz ilość potrzebnej pamięci. Rozpatruje przypadki optymistyczny, pesymistyczny i średni.\\[0.3cm]\pause 
        
        Skupia się raczej na dużych i bardzo dużych danych wejściowych. 
\end{frame}

\begin{frame}{Notacja dużego O.}
    Notacja Big O jest notacją matematyczną, która opisuje zachowanie funkcji w jej granicy, czyli, gdy argument rośnie do nieskończoności.\\[0.3cm]\pause 
    
    Przykładowo dla funkcji wielomianowych, zachowanie w nieskończoności będzie takie, jak stopień wielomianu, nie zważając na wartości jego współczynników.\pause 

    \begin{center}
        \begin{tabular}{ |c | c | } 
        \hline
        Liczba milisekund potrzebna do wykonania algorytmu & Złożoność Obliczeniowa \\ 
        \hline \pause 
        $5N^2 + 100$ & $O(N^2)$ \\ 
        \hline \pause
        
        $0.000001N^2 + 1N - 1$ & $O(N^2)$ \\ 
        \hline \pause
        $100N^2 - 10000000000n$ & $O(N^2)$ \\ 
        \hline \pause
        $5N + 100$ & $O(N)$ \\ 
        \hline
        \end{tabular}
    \end{center}

        
\end{frame}

\begin{frame}{Notacja dużego O.}
    Patrzymy na najwyższy stopień wielomianu.

    \begin{center}
        \begin{tabular}{ |c | c | } 
        \hline
        Liczba milisekund potrzebna do wykonania algorytmu & Złożoność Obliczeniowa \\ 
        \hline 
        $5\mathbf{N^2} + 100$ & $O(N^2)$ \\ 
        \hline
        
        $0.000001\mathbf{N^2} + 1N - 1$ & $O(N^2)$ \\ 
        \hline 
        $100\mathbf{N^2} - 10000000000n$ & $O(N^2)$ \\ 
        \hline 
        $5\mathbf{N} + 100$ & $O(N)$ \\ 
        \hline
        \end{tabular}
    \end{center}
    
    Ostatni algorytm będzie \textbf{asymptotycznie} najszybszy.
        
\end{frame}

\end{document}